Antecedentes de derecho
Ley 21.442 artículo 21 párrafo 1, 2 y 3: “El administrador estará obligado a rendir cuenta documentada y pormenorizada de su gestión, ante el comité de administración en forma mensual y ante la asamblea de copropietarios en cada sesión ordinaria y al término de su administración.
El administrador deberá consignar, en cada cuenta que rinda, el detalle de los ingresos y gastos, incluida las remuneraciones y pagos relativos a seguridad social del personal contratado, así como el saldo de caja, entregando una copia informada por el banco de todas las cuentas bancarias, cartolas de estas cuentas y respaldo de pago de los seguros contratados.
Para estos efectos, la documentación correspondiente deberá estar a disposición de los copropietarios y arrendatarios del condominio y ser proporcionada con, al menos, veinticuatro horas de antelación respecto de las sesiones ordinarias de la asamblea de copropietarios o de las reuniones del comité de administración en que deba rendirse la cuenta mensual”
Ley 21.442 artículo 44: “Serán de competencia de los juzgados de policía local correspondientes y se sujetarán a las disposiciones de la ley Nº 18.287 y, en subsidio, a las normas del Libro Primero del Código de Procedimiento Civil, las contiendas que surjan en el ámbito del régimen especial de copropiedad inmobiliaria establecido en esta ley y que se promuevan entre los copropietarios o entre éstos y la asamblea de copropietarios, el comité de administración o el administrador, o entre estos mismos órganos de administración de la copropiedad inmobiliaria, relativas a la administración o funcionamiento del condominio, para lo cual estos tribunales estarán investidos de todas las facultades que sean necesarias a fin de resolver esas controversias. En el ejercicio de estas facultades” [...]
Código de Procedimiento Civil Libro I Título V artículo 30: “Los escritos y documentos se presentarán por vía electrónica conforme se dispone en los artículos 5º y 6º, respectivamente, de la Ley General sobre Tramitación Electrónica de los Procedimientos Judiciales.”
Ley 21.442 artículo 44 letra b, c: “b) Declarar la nulidad de los acuerdos adoptados por la asamblea con infracción de las normas de esta ley y de su reglamento o de los reglamentos de copropiedad. Para estos efectos, el tribunal deberá sujetarse a lo dispuesto en el inciso quinto del artículo 10 de esta ley.
c) Citar a asamblea de copropietarios, si el administrador o el presidente del comité de administración no lo hicieren, aplicándose al efecto las normas contenidas en el artículo 654 del Código de Procedimiento Civil, en lo que fuere pertinente. A esta asamblea deberá asistir un notario como ministro de fe, quien levantará acta de lo actuado. La citación a asamblea se notificará mediante carta certificada y/o correo electrónico, sujetándose a lo previsto en el inciso primero del artículo 16 de la presente ley. Para estos efectos, el administrador, a requerimiento del juez, deberá poner a disposición del tribunal la nómina de copropietarios a que se refiere el citado inciso primero, dentro de los cinco días siguientes desde que le fuere solicitada y, si así no lo hiciere, se le aplicará la multa prevista en el inciso tercero del artículo 27.”
Ley 21.442 artículo 15 antepenúltimo y penúltimo párrafo: “El presidente del comité de administración, o quien la asamblea designe, deberá levantar acta de las sesiones y de las consultas por escrito efectuadas. En ellas se deberá dejar constancia de los acuerdos adoptados, especificando el quórum de constitución de la sesión y de adopción de los acuerdos.
Las actas deberán constar en un libro de actas foliado, ya sea en formato papel o digital, que asegure su respaldo fehaciente, y ser firmadas de forma presencial o electrónica, a más tardar dentro de los treinta días siguientes a la adopción del acuerdo, por todos los miembros del comité de administración o por los copropietarios que la asamblea designe, quedando el libro de actas y todos los antecedentes que respalden los acuerdos bajo custodia del presidente de dicho comité, sea que se trate de documentos impresos, digitales, audiovisuales o en otros formatos. La infracción a estas obligaciones será sancionada con multa de una a tres unidades tributarias mensuales, la que se duplicará en caso de reincidencia o falta de subsanación.”
Ley 21.442 artículo 16: “El comité de administración, a través de su presidente, o si éste no lo hiciere, del administrador, deberá citar a asamblea a todos los copropietarios o representantes, personalmente o mediante carta certificada dirigida al domicilio o a través de correo electrónico que, para estos efectos, estuvieren incorporados en el registro de copropietarios, o en la secretaría municipal cuando se trate de condominios de viviendas sociales. Esta citación se cursará con una anticipación mínima de cinco días y que no exceda de quince. Si no hubieren registrado un domicilio o correo electrónico, se entenderá para todos los efectos que tienen su domicilio en la respectiva unidad del condominio. El administrador deberá mantener actualizado el registro de copropietarios del condominio, debiendo velar por la protección y resguardo de los datos personales.”
Código Penal Artículo 470 n° 5: “A los que cometieren defraudaciones sustrayendo, ocultando, destruyendo o inutilizando en todo o en parte algún proceso, expediente, documento u otro papel de cualquiera clase.”
Ley 21.442 Artículo 7: “Cada copropietario deberá pagar las obligaciones económicas del condominio dentro de los diez primeros días siguientes a la fecha de emisión del correspondiente aviso de cobro, salvo que el reglamento de copropiedad establezca otra periodicidad o plazo. Si incurriere en mora, la deuda devengará el interés que se disponga en dicho reglamento, o en su defecto en el reglamento tipo, el que no podrá ser superior al 50% del interés corriente bancario.”
Ley 21.442 Artículo 20 n° 9: “Suspender o requerir la suspensión, según sea el caso, y con acuerdo del comité de administración, del servicio eléctrico, de telecomunicaciones o de calefacción que se suministra a aquellas unidades cuyos propietarios se encuentren morosos en el pago de tres o más cuotas, continuas o discontinuas, de los gastos comunes.”